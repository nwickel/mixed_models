\documentclass{beamer}
% \usepackage{pgfpages}
% \pgfpagesuselayout{4 on 1}[a4paper,landscape,border shrink=5mm]
\usepackage{tikz}
\usetikzlibrary{arrows, positioning}
\usepackage{listings}
\usepackage[utf8,latin1]{inputenc}
\usepackage[natbibapa]{apacite}
\usepackage{pgfplots}
\makeatletter \def\newblock{\beamer@newblock} \makeatother  

\beamertemplatenavigationsymbolsempty
\setbeamertemplate{itemize items}[circle]
\setbeamertemplate{section in toc}[circle]
%\mode<beamer>{\setbeamercolor{math text displayed}{fg=iwmorange!50!black}}
\setbeamercolor{block body}{bg=iwmorange!50!white}
\setbeamercolor{block title}{fg=white, bg=iwmorange}

\definecolor{iwmorange}{RGB}{255,105,0}
\definecolor{iwmgrau}{RGB}{67,79,79}
\setbeamercolor{title}{fg=iwmorange}
\setbeamercolor{frametitle}{fg=iwmorange}
\setbeamercolor{structure}{fg=iwmorange}
\setbeamercolor{normal text}{fg=iwmgrau}
\setbeamercolor{author}{fg=iwmgrau}
\setbeamercolor{date}{fg=iwmgrau}
\color{white}

\title{Power simulation for linear mixed-effects models}
\author{Nora Umbach%\footnote{These slides are a modified version of slides created by \url{https://osf.io/ /}. }
}
%\institute{\includegraphics[scale=.15]{figures/ut_logo}}
\date{July 26, 2021}

\newcommand{\vect}[1]{\mathbf{#1}}
\newcommand{\mat}[1]{\mathbf{#1}}
\newcommand{\gvect}[1]{\boldsymbol{#1}}
\newcommand{\gmat}[1]{\boldsymbol{#1}}

\lstset{language=R,%
  literate={Ü}{{\"U}}1
           {ü}{{\"u}}1,
  %backgroundcolor=\color{iwmgrau!80!white},
  basicstyle=\ttfamily\color{iwmorange},
  frame=single,
  commentstyle=\slshape\color{black},
  keywordstyle=\bfseries\color{white},
  identifierstyle=\color{white},
  stringstyle=\color{green!85!black},
  numbers=none,%numberstyle=\tiny,
  basewidth={.5em, .4em},
  showstringspaces=false,
  emphstyle=\color{red!50!white}}

\lstdefinestyle{plain}{language=R,
  frame=none,
  basicstyle=\ttfamily\color{iwmorange},
  commentstyle=\slshape\color{iwmgrau},
  keywordstyle=\bfseries\color{iwmgrau},
  identifierstyle=\color{iwmgrau},
  stringstyle=\color{iwmgrau},
  numbers=none,
  basewidth={.5em, .4em},
  showstringspaces=false}

\pgfmathdeclarefunction{gauss}{2}{%
  \pgfmathparse{1/(#2*sqrt(2*pi))*exp(-((x-#1)^2)/(2*#2^2))}%
}

\AtBeginSection[]{
  \frame{
    \tableofcontents[sectionstyle=show/hide, subsectionstyle=show/show/hide]}}

\setbeamertemplate{headline}{
 \begin{beamercolorbox}{section in head}
   \vskip5pt\insertsectionnavigationhorizontal{\paperwidth}{}{}\vskip2pt
 \end{beamercolorbox}
}

\setbeamertemplate{footline}{\vskip-2pt\hfill\insertframenumber$\;$\vskip2pt}

\begin{document}

\begin{frame}
\thispagestyle{empty}
\titlepage
\end{frame}

\begin{frame}{Linear mixed-effects models}

{\tiny
\[
  \begin{pmatrix}
    y_{11}\\
    \vdots\\
    y_{ij}\\
    \vdots\\
    y_{qn}
  \end{pmatrix}
  =
  \begin{pmatrix}
    1 & x_{111} & x_{121} & \dots & x_{1p1}\\
    1 & x_{112} & x_{122} & \dots & x_{1p2}\\
    \vdots & \vdots & \vdots && \vdots \\
    1 & x_{211} & x_{221} & \dots & x_{2p1}\\
    \vdots & \vdots & \vdots && \vdots \\
    1 & x_{n1q} & x_{n2q} & \dots & x_{npq}\\
  \end{pmatrix}
\times
  \begin{pmatrix}
    \beta_0\\
    \beta_1\\
    \vdots\\
    \beta_p
  \end{pmatrix}
+
  \begin{pmatrix}
    1 & 0 & \dots & 0 & 0\\
    \vdots & \vdots && \vdots & \vdots \\
    0 & 1 & \dots & 0 & 0\\
    \vdots & \vdots && \vdots & \vdots \\
    0 & 0 & \dots & 1 & 0\\
    \vdots & \vdots && \vdots & \vdots \\
    0 & 0 & \dots & 0 & 1
  \end{pmatrix}
\times
  \begin{pmatrix}
    \upsilon_1\\
    \vdots\\
    \upsilon_{qn}\\
  \end{pmatrix}
+
  \begin{pmatrix}
    e_{11}\\
    \vdots\\
    e_{ij}\\
    \vdots\\
    e_{qn}
  \end{pmatrix}
\]
}
and the corresponding vector equation
\[
  \mat{y} = \mat{X} \, \gmat{\beta} + \mat{Z} \,\gmat{\upsilon} + \mat{e}
\]
\
\end{frame}

\begin{frame}{Outline}
\tableofcontents
\end{frame}

\section{Simple regression}

\begin{frame}{Example: \citet{Fox2008}}
  \begin{itemize}
    \item Vocabulary Data from the U.S.\ General Social Surveys
    \item $N = \text{21,638}$
    \item Outcome:
      \begin{itemize}
        \item Vocabulary score
      \end{itemize}
    \item Predictor:
      \begin{itemize}
        \item Years of education
      \end{itemize}
  \end{itemize}
\end{frame}

\begin{frame}{Example: \citet{Fox2008}}
\begin{columns}
  \column{.6\textwidth}
  \includegraphics[scale=.8]{figures/voc}
  \column{.4\textwidth}
  \begin{align*}
    y_{i} & = \beta_0 + \beta_1 x_{i} + \varepsilon_i\\
    \varepsilon_i & \sim N(0, \sigma^2)
  \end{align*}
  \begin{tabular}{lr}
    \hline
         &  Mean (SD) \\
    \hline
    Vocabulary &  6.00 (2.17)\\
    Education  & 12.80 (3.04)\\
    \hline
  \end{tabular}
\end{columns}
\end{frame}

\begin{frame}{}
  \begin{block}{Exercise}
    \begin{itemize} 
      \item What are good assumptions about the distributions of
        \begin{itemize}
          \item Vocabulary
          \item Education
        \end{itemize}
    \end{itemize}
  \end{block}
\end{frame}

{\setbeamercolor{background canvas}{bg=iwmgrau!80!white}

\begin{frame}[fragile]{Vocabulary data}
  \begin{lstlisting}
dat <- read.table("Vocabulary.txt", header=TRUE, 
                  stringsAsFactors=TRUE)

lm1 <- lm(vocabulary ~ education, dat)
summary(lm1)

plot(jitter(vocabulary, 2) ~ jitter(education, 2),
     dat, pch=".", xlab="Years of education",
     ylab="Vocabulary score", col="darkgrey")
abline(lm1)
  \end{lstlisting}
\end{frame}

\begin{frame}[fragile]{Power simulation}
  \begin{lstlisting}
n <- 150

prob <- table(dat$education) / nrow(dat)
simdat <- data.frame(x1 = sample(0:20, n, TRUE, prob))

pval <- replicate(2000, {
        y <- 1.5 + 0.2*simdat$x1 + rnorm(n, sd=2.2)
        m1 <- lm(y ~ 1, simdat)
        m2 <- lm(y ~ x1, simdat)
        anova(m1, m2)$"Pr(>F)"[2]
})
mean(pval < 0.05)
  \end{lstlisting}
\end{frame}

}

\section{Mixed-effects regression}

\begin{frame}{Example: \citet{Bauer2005}}
  \begin{itemize}
    \item $N = \text{7,185}$
    \item 160 schools
    \item Outcome:
      \begin{itemize}
        \item math achievement ($\bar y = 12.75, sd = 6.88$)
      \end{itemize}
    \item Predictors:
      \begin{enumerate}
        \item child SES (metric, group mean centered with $sd = .66$)
        \item school sector ($0 = \text{public school}$, $1 = \text{private
          school}$, 49\,\% of schools were private)
        % \item disciplinary climate (metric, high values reflect greater
        %   disciplinary problems, grand mean centered with $sd = .94$)
      \end{enumerate}
  \end{itemize}
\end{frame}

\begin{frame}{Statistical model}
  \begin{align*}
    y_{ij} = & \beta_{0} + \beta_{1}cses_{ij} + \beta_{2}meanses_{i} +
    \beta_{3}sector_{i}\\
    & + \beta_{4}(cses_{ij} \times meanses_i) +
    \beta_5(cses_{ij}\times sector_i)\\
    & + \upsilon_{0i} + \upsilon_{1i}cses_{ij} + \varepsilon_{ij}
  \end{align*}
  \[
    \begin{pmatrix}
      \upsilon_{0i} \\
      \upsilon_{1i}
    \end{pmatrix}
    \sim N\left(
    \begin{pmatrix}
      0 \\
      0
    \end{pmatrix},
    \gmat{\Sigma}_{\upsilon} = 
    \begin{pmatrix}
      \sigma_{\upsilon_{0}}^2 & \sigma_{\upsilon_{0}\upsilon_1} \\
      \sigma_{\upsilon_{0}\upsilon_1} & \sigma_{\upsilon_{1}}^2
    \end{pmatrix}
    \right) \text{i.i.d.},~~~\varepsilon_i \sim N(\vect{0}, \sigma^2
    \mat{I}_{n_i})~\text{i.i.d.}
  \]
  \[
    i = 1,\dots,I, ~~~ j=1,\dots,n_i
  \]

\end{frame}

{\setbeamercolor{background canvas}{bg=iwmgrau!80!white}

\begin{frame}[fragile]{Highschool and beyond data}
  \begin{lstlisting}
library(lme4)
dat <- read.table("hsbdataset.txt", header=TRUE)

# Look at standard deviations
sd(dat$ses)
sd(dat$cses)
table(dat$sector) / nrow(dat)

lme1 <- lmer(mathach ~ cses*meanses + cses*sector + 
            (cses | school), dat)
summary(lme1)

library(lattice)
xyplot(mathach ~ cses | as.factor(sector), dat, 
       groups=school, type="r")
  \end{lstlisting}
\end{frame}


\begin{frame}[fragile]{Simulate data frame}
  \begin{lstlisting}
nschool <- 150
nstudent <- 40
simdat <- data.frame( 
                  id = seq_len(nstudent),
                  school = seq_len(nschool),
                  sector = rep(0:1, each=nschool/2),
                  ses = rnorm(nschool*nstudent, 0, .8)
)

# Mean centering ses
simdat$meanses <- ave(simdat$ses, simdat$school)
simdat$cses <- simdat$ses - simdat$meanses
  \end{lstlisting}
\end{frame}

\begin{frame}[fragile]{Define parameters}
  \begin{lstlisting}
se  <- 6.5
sp  <- 1.6
spw <- 0.3
cov <- 0.4*sp*spw

fix <- c("(Intercept)"=12, meanses=5, cses=2.5, 
         sector=1, "cses:meanses"=0.7, 
         "cses:sector"=-0.7)

mat <- matrix(c(sp^2, cov,
                cov, spw^2), 2, 2)
ran <- chol(mat) / se

params <- list(theta=t(ran)[lower.tri(t(ran), TRUE)],
               beta=beta, sigma=se)
  \end{lstlisting}
\end{frame}


\begin{frame}[fragile]{Power simulation}
  \begin{lstlisting}
pval <- replicate(200, {
  mathach <- simulate(~ cses*meanses + cses*sector + 
                     (cses | school),
                     newparams=params, newdata=simdat,
                     family=gaussian)$sim_1
  m1 <- lmer(mathach ~ meanses + cses + sector
             + (cses | school), simdat, REML=FALSE)
  m2 <- lmer(mathach ~ cses*meanses + cses*sector
             + (cses | school), simdat, REML=FALSE)
  anova(m1, m2)$"Pr(>Chisq)"[2]
})

mean(pval < 0.05)
hist(pval)
  \end{lstlisting}
\end{frame}

}

\appendix
%\begin{frame}[allowframebreaks]{References}
\begin{frame}{References}
\renewcommand{\bibfont}{\footnotesize}
\bibliographystyle{apacite}
\bibliography{../../../literature/nu}
\vfill
\end{frame}

\end{document}

