\documentclass[aspectratio=169]{beamer}

\usepackage{listings}
\usepackage[utf8,latin1]{inputenc}
\usepackage[style = apa, backend = biber, natbib = true]{biblatex}
\addbibresource{../../literature/lit.bib}

\makeatletter \def\newblock{\beamer@newblock} \makeatother

\beamertemplatenavigationsymbolsempty
\setbeamertemplate{itemize items}[circle]
\setbeamertemplate{section in toc}[circle]
\mode<beamer>{\setbeamercolor{math text displayed}{fg=iwmgray}}
\setbeamercolor{block body}{bg=iwmorange!50!white}
\setbeamercolor{block title}{fg=white, bg=iwmorange}

% Definitions for biblatex
\setbeamercolor{bibliography entry note}{fg=iwmgray}
\setbeamercolor{bibliography entry author}{fg=iwmgray}
\setbeamertemplate{bibliography item}{}

\definecolor{iwmorange}{RGB}{255,105,0}
\definecolor{iwmgray}{RGB}{67,79,79}
\definecolor{iwmblue}{RGB}{60,180,220}
\definecolor{iwmgreen}{RGB}{145,200,110}
\definecolor{iwmpurple}{RGB}{120,0,75}

\setbeamercolor{title}{fg=iwmorange}
\setbeamercolor{frametitle}{fg=iwmorange}
\setbeamercolor{structure}{fg=iwmorange}
\setbeamercolor{normal text}{fg=iwmgray}
\setbeamercolor{author}{fg=iwmgray}
\setbeamercolor{date}{fg=iwmgray}

\lstset{language = R,%
  basicstyle = \ttfamily\color{iwmgray},
  frame = single,
  rulecolor = \color{iwmgray},
  commentstyle = \slshape\color{iwmgreen},
  keywordstyle = \bfseries\color{iwmgray},
  identifierstyle = \color{iwmpurple},
  stringstyle = \color{iwmblue},
  numbers = none,%left,numberstyle = \tiny,
  basewidth = {.5em, .4em},
  showstringspaces = false,
  emphstyle = \color{red!50!white}}

\title{Generalized linear mixed-effects models}
\author{Nora Wickelmaier}
\date{Last modified: \today}

\AtBeginSection[]{
  \frame{
    \tableofcontents[sectionstyle=show/hide, subsectionstyle=show/show/hide]}}

% \setbeamertemplate{headline}{
%  \begin{beamercolorbox}{section in head}
%    \vskip5pt\insertsectionnavigationhorizontal{\paperwidth}{}{}\vskip2pt
%  \end{beamercolorbox}
% }

\setbeamertemplate{footline}{\vskip-2pt\hfill\insertframenumber$\;$\vskip2pt}

\begin{document}

\begin{frame}{}
\thispagestyle{empty}
\titlepage
\end{frame}

% \begin{frame}{Outline}
% \tableofcontents
% \end{frame}

\begin{frame}[fragile]{GLMMs}
  \begin{itemize}
    \item Like linear models, mixed-effects models can be extended so that they
      allow for response variables that have arbitrary distributions
    \item A GLM(M) is a specific combination of a response distribution, a link
      function, and a linear predictor 
    \item We can choose (almost) all the link functions that work with
      \texttt{glm()} for \texttt{glmer()}
  \end{itemize}
\begin{lstlisting}
## Family name       Link functions
   binomial          logit, probit, log, cloglog
   gaussian          identity, log, inverse
   Gamma             identity, inverse, log
   inverse.gaussian  1/mu^2, identity, inverse, log
   poisson           log, identity, sqrt
\end{lstlisting}
\end{frame}

\begin{frame}{Example: }
\end{frame}

\appendix
%\begin{frame}[allowframebreaks]{References}
\begin{frame}{References}
  \printbibliography
\end{frame}

\end{document}

