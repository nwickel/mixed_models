\documentclass[aspectratio=169]{beamer}

\usepackage{listings}
%\usepackage[utf8]{inputenc}
\usepackage[style = apa, backend = biber, natbib = true]{biblatex}
\addbibresource{../../literature/lit.bib}

\usepackage{fancyvrb}
\usepackage{fontawesome5}                % get icons
\usepackage{multirow}
\usepackage{color, colortbl}

\usepackage{tikz}
\usetikzlibrary{fit}
\usepackage[edges]{forest}

\lstset{language = R,%
  basicstyle = \ttfamily\color{iwmgray},
  frame = single,
  rulecolor = \color{iwmgray},
  commentstyle = \slshape\color{iwmgreen},
  keywordstyle = \bfseries\color{iwmgray},
  identifierstyle = \color{iwmpurple},
  stringstyle = \color{iwmblue},
  numbers = none,%left,numberstyle = \tiny,
  basewidth = {.5em, .4em},
  showstringspaces = false,
  emphstyle = \color{red!50!white}}

\makeatletter \def\newblock{\beamer@newblock} \makeatother

\beamertemplatenavigationsymbolsempty
\setbeamertemplate{itemize items}[circle]
\setbeamertemplate{section in toc}[circle]
\mode<beamer>{\setbeamercolor{math text displayed}{fg=iwmgray}}
\setbeamercolor{block body}{bg=iwmorange!50!white}
\setbeamercolor{block title}{fg=white, bg=iwmorange}

% Definitions for biblatex
\setbeamercolor{bibliography entry note}{fg=iwmgray}
\setbeamercolor{bibliography entry author}{fg=iwmgray}
\setbeamertemplate{bibliography item}{}

\definecolor{iwmorange}{RGB}{255,105,0}
\definecolor{iwmgray}{RGB}{67,79,79}
\definecolor{iwmblue}{RGB}{60,180,220}
\definecolor{iwmgreen}{RGB}{145,200,110}
\definecolor{iwmpurple}{RGB}{120,0,75}

\setbeamercolor{title}{fg=iwmorange}
\setbeamercolor{frametitle}{fg=iwmorange}
\setbeamercolor{structure}{fg=iwmorange}
\setbeamercolor{normal text}{fg=iwmgray}
\setbeamercolor{author}{fg=iwmgray}
\setbeamercolor{date}{fg=iwmgray}

\newcommand{\vect}[1]{\mathbf{#1}}
\newcommand{\mat}[1]{\mathbf{#1}}
\newcommand{\gvect}[1]{\boldsymbol{#1}}
\newcommand{\gmat}[1]{\boldsymbol{#1}}

\AtBeginSection[]{
  \frame{
    \tableofcontents[sectionstyle=show/hide, subsectionstyle=show/show/hide]}}

\setbeamertemplate{headline}{
 \begin{beamercolorbox}{section in head}
   \vskip5pt\insertsectionnavigationhorizontal{\paperwidth}{}{}\vskip2pt
 \end{beamercolorbox}
}

\setbeamertemplate{footline}{\vskip-2pt\hfill\insertframenumber$\;$\vskip2pt}

\title{Programming in R}
\author{Nora Wickelmaier}
\date{Last modified: \today}

\begin{document}

\begin{frame}{}
\thispagestyle{empty}
\titlepage
\end{frame}

\begin{frame}[fragile]{Programming resources}
  \footnotesize
  \begin{tabular}{ll}
    Learning statistics with R  & {\url{https://learningstatisticswithr.com/book/}} \\
    &\\
    R for Data Science & {\url{https://r4ds.hadley.nz/}} \\
    &\\
    Advanced R  & {\url{https://adv-r.hadley.nz/}} \\
    &\\
    Happy Git and GitHub for the useR & {\url{https://happygitwithr.com/}} \\
    &\\
    R Programming for Research & {\url{https://geanders.github.io/RProgrammingForResearch/}} \\
    &\\
    Building reproducible analytical pipelines with R & {\url{https://raps-with-r.dev/}} \\
    &\\
    Data Skills for Reproducible Science & {\url{https://psyteachr.github.io/msc-data-skills/}} \\
  \end{tabular}
\end{frame}

\section{Style guidelines}

\begin{frame}[<+->]{Style guidelines in R}
  \begin{itemize}
    \item R has no mandatory or commonly accepted style guide
    \item However, Hadley Wickham and Google developed style guides which are
      now widely accepted
      \begin{itemize}
        \item \url{https://google.github.io/styleguide/Rguide.html}
        \item \url{https://style.tidyverse.org/}
      \end{itemize}
    \item It is always a good idea to follow a style guide and not ``create''
      your own rules (if you deviate, be consistent!)
    \item A style guide helps with
      \begin{itemize}
        \item Keeping code clean which is easier to read and interpret
        \item Making it easier to catch and fix mistakes
        \item Making it easier for others to follow and adapt your code
        \item Preventing possible problems, e.\,g., avoiding dots in function
          names
      \end{itemize}
  \end{itemize}
  \nocite{Wickham_styleguide, Anderson2023}
\end{frame}

\begin{frame}[fragile, allowframebreaks]{File names}
  \begin{itemize}
    \item File names should be meaningful and end in .R
    \item Avoid using special characters in file names
    \item Stick with numbers, letters, \verb+-+, and \verb+_+
  \begin{lstlisting}[identifierstyle = \bfseries\color{iwmgray}]
# Good
fit_models.R
utility_functions.R

# Bad
fit models.R
foo.r
stuff.r
  \end{lstlisting}
      \framebreak

    \item If files should be run in a particular order, prefix them with numbers
    \item If it seems likely you’ll have more than 10 files, left pad with zero
  \begin{lstlisting}[identifierstyle = \bfseries\color{iwmgray}]
00_download.R
01_explore.R
...
09_model.R
10_visualize.R
  \end{lstlisting}
    \item If you later realize that you missed some steps, it’s tempting to use
      02a, 02b, etc.
    \item However, it is generally better to bite the bullet and rename all
      files
  \end{itemize}
\end{frame}

\begin{frame}[fragile, allowframebreaks]{Object names}
  \begin{itemize}
    \item Variable and function names should use only lowercase letters,
      numbers, and \verb+_+
    \item Use underscores (\verb+_+) (so called snake case) to separate words
      within a name
  \begin{lstlisting}[identifierstyle = \bfseries\color{iwmgray}]
# Good
day_one
day_1

# Bad
DayOne
dayone
  \end{lstlisting}
  \framebreak

    \item Generally, variable names should be nouns and function names should be
      verbs
    \item Strive for names that are concise and meaningful
      \begin{lstlisting}[identifierstyle = \bfseries\color{iwmgray}]
# Good
day_one

# Bad
first_day_of_the_month
djm1
  \end{lstlisting}
  \framebreak

    \item Avoid re-using names of common functions and variables
  \begin{lstlisting}
# Bad
T <- FALSE
c <- 10
mean <- function(x) sum(x)
  \end{lstlisting}
  \end{itemize}
\end{frame}

\begin{frame}[fragile, allowframebreaks]{Spacing}
  \begin{itemize}
    \item Always put a space after a comma, never before
  \begin{lstlisting}
# Good
x[, 1]

# Bad
x[,1]
x[ ,1]
x[ , 1]
  \end{lstlisting}
  \framebreak

  \item Do not put spaces inside or outside parentheses for regular function
    calls
  \begin{lstlisting}
# Good
mean(x, na.rm = TRUE)

# Bad
mean (x, na.rm = TRUE)
mean( x, na.rm = TRUE )
  \end{lstlisting}
  \framebreak

\item Place a space before and after \texttt{()} when used with \texttt{if},
  \texttt{for}, or \texttt{while}
  \begin{lstlisting}
# Good
if (debug) {
  show(x)
}

# Bad
if(debug){
  show(x)
}
  \end{lstlisting}
  \framebreak

\item Place a space after \texttt{()} used for function arguments
  \begin{lstlisting}
# Good
function(x) {}

# Bad
function (x) {}
function(x){}
  \end{lstlisting}
  \framebreak

    \item Most infix operators (\verb+==+, \verb|+|, \verb+-+, \verb+<-+, etc.)
      should always be surrounded by spaces
  \begin{lstlisting}
# Good
height <- (feet * 12) + inches
mean(x, na.rm = TRUE)

# Bad
height<-feet*12+inches
mean(x, na.rm=TRUE)
  \end{lstlisting}
  \framebreak

    \item There are a few exceptions, which should never be surrounded by
      spaces: \verb+::+, \verb+:::+, \verb+$+, \verb+@+, \verb+[+, \verb+[[+,
      \verb+?+, \verb+^+, and \verb+:+
      {\small
  \begin{lstlisting}
# Good
sqrt(x^2 + y^2)
df$z
x <- 1:10
package?stats
?mean

# Bad
sqrt(x ^ 2 + y ^ 2)
df $ z
x <- 1 : 10
package ? stats
? mean
  \end{lstlisting}
  }
    \item Adding extra spaces is ok if it improves alignment of \verb+=+ or
      \verb+<-+
  \begin{lstlisting}
# Good
list(
  total = a + b + c,
  mean  = (a + b + c) / n
)

# Also fine
list(
  total = a + b + c,
  mean = (a + b + c) / n
)
  \end{lstlisting}
  \end{itemize}
\end{frame}

% CITE:
% https://style.tidyverse.org/index.html
% R Programming for Reserach: https://geanders.github.io/RProgrammingForResearch/ 
% Building reproducible analytical pipelines with R: https://raps-with-r.dev/

\section{Script organisation}

\begin{frame}[fragile]{Script header}
  \begin{itemize}
    \item It can be very helpful to have some general information right at the
      top when opening a script
      \begin{lstlisting}
# 01_preprocessing.R
#
# Cleaning up toy data set (Methods Seminar SS2024)
#
# Input:  rawdata/RDM_MS_SS2024_download_2024-06-07.csv
# Output: processed/data_rdm-ms-ss2024_cleaned.csv
#         processed/data_rdm-ms-ss2024_cleaned.RData
#
# Created: 2024-06-03, NW
      \end{lstlisting}
    \item These metadata help you remember faster what you did
    \item Might not be necessary when using consistent version control (but does
      not hurt either)
  \end{itemize}
\end{frame}

\begin{frame}[fragile]{Line length}
  {}
  \begin{center}
    {\Large\bf Keep lines to 80 characters or less!}
  \end{center}
  \begin{lstlisting}
# Good
my_df <- data.frame(n = 1:3,
                    letter = c("a", "b", "c"),
                    cap_letter = c("A", "B", "C"))

# Bad
my_df <- data.frame(n = 1:3, letter = c("a", "b", "c"), cap_letter = c("A", "B", "C"))
  \end{lstlisting}
  \begin{itemize}
    \item Ensures that your code is formatted in a way that you can see all of
      the code without scrolling horizontally
    \item To set your script pane to be limited to 80 characters, go to\\
      \verb+RStudio -> Preferences -> Code -> Display+\\
      and set ``Margin Column'' to 80
  \end{itemize}
\end{frame}

\begin{frame}[fragile, allowframebreaks]{File organisation}
  \begin{itemize}
    \item Try to write scripts that are concerned with one (major) task
    \item If you can find a name, that captures the content, it is usually a
      good way to start
    \item Some (random) examples
\begin{lstlisting}[identifierstyle = \bfseries\color{iwmgray}]
download-data.R
data-cleaning.R
cluster_analysis_exp1.R
visualization_logistic-model.R
anova_h1.R
  \end{lstlisting}
  \framebreak

    \item Export data sets for new scripts (do not make yourself run all scripts
      up to script 5 each time, just because you need the data in a certain
      format)
      \begin{lstlisting}
# Interoperable
write.table(dat,
            file = "data_exp1_cleaned.csv",
            sep = ";",
            quote = FALSE,
            row.names = FALSE)

# Preserve order of factor levels, date formats, etc.
save(dat, file = "data_exp1_cleaned.RData")
      \end{lstlisting}
  \end{itemize}
\end{frame}

\begin{frame}[fragile, allowframebreaks]{Internal structure}
  \begin{itemize}
    \item Use commented lines with \texttt{-} or \texttt{=} to break your file
      up into chunks
    \item Load additional packages at the beginning of the script
  \begin{lstlisting}
library(lme4)
library(sjPlot)

# Load data ---------------------------

# Plot data ---------------------------
  \end{lstlisting}
  \framebreak

    \item If you load several packages, be aware that the order of loading
      matters!
    \item If you use only one or two functions from a package, get the function
      with \verb+::+ instead of loading the whole package
  \begin{lstlisting}
library(lme4)
...

# Fit mixed-effects model to test Hypothesis 1
lme1 <- lmer(Reaction ~ Days + (Days | Subject), sleepstudy)
summary(lme1)
sjPlot::tab_model(lme1)
  \end{lstlisting}
  \framebreak

    \item Group related pieces of code together
    \item Separate blocks of code by empty spaces
  \begin{lstlisting}
# Load data
library(faraway)
data(nepali)

# Relabel sex variable
nepali$sex <- factor(nepali$sex, 
                     levels = c(1, 2),
                     labels = c("Male", "Female"))
  \end{lstlisting}
  \end{itemize}
\end{frame}


\section[Functions]{Writing functions}

\begin{frame}{Functions and arguments}
  \begin{itemize}
    \item Functions in R consist of
    \begin{itemize}
    \item a name
    \item a pair of brackets
    \item the arguments (none, one, or more)
    \item a return value (visible, invisible, \texttt{NULL})
    \end{itemize}
    \item Arguments are passed
    \begin{itemize}
      \item either without name (in the defined order)\\
        $\to$ positional matching
      \item or with name (in arbitrary order)\\
        $\to$ keyword matching
    \end{itemize}
    \item Even if no arguments are passed, the brackets need to be written,
      e.\,g., \texttt{ls()}, \texttt{dir()}, \texttt{getwd()}
    \item Entering only the name of a function without brackets will
      display the R code of that function
  \end{itemize}
\end{frame}

\begin{frame}{Writing functions}
  \begin{itemize}
    \item Let us implement a simple function ourselves
    \item A function that implements a two-sample $t$ test
      \begin{equation*}
  T = \frac{\bar{x} - \bar{y}}
      {\sqrt{\hat{\sigma}^2\, \left(\frac{1}{n} + \frac{1}{m}\right)}}
    = \frac{\bar{x} - \bar{y}}
      {\sqrt{\frac{(n-1) \, s_x^2 + (m-1) \, s_y^2}
          {n+m-2}\cdot \left(\frac{1}{n} + \frac{1}{m}\right)}}
      \end{equation*}
      with
      \begin{equation*}
        T \sim t(n+m-2)
      \end{equation*}
  \end{itemize}
\end{frame}

\begin{frame}[fragile]{Writing functions}
\begin{lstlisting}
# Example: a handmade t test function
twosam <- function(y1, y2){              # definition
  n1  <- length(y1); n2  <- length(y2)   # body
  yb1 <- mean(y1);   yb2 <- mean(y2)
  s1  <- var(y1);    s2  <- var(y2)
  s   <- ((n1 - 1)*s1 + (n2 - 1)*s2)/(n1 + n2 - 2)
  tst <- (yb1 - yb2)/sqrt(s*(1/n1 + 1/n2))
  tst             # return value, can also be a list
}

# Calling the function
tstat <- twosam(PlantGrowth$weight[PlantGrowth$group == "ctrl"],
                PlantGrowth$weight[PlantGrowth$group == "trt1"])
tstat
\end{lstlisting}
  \nocite{Venables2022}
\end{frame}

\begin{frame}[fragile]{Named arguments}
  \begin{itemize}
    \item If there is a function \texttt{fun1} defined by

\begin{lstlisting}
  fun1 <- function(data, data.frame, graph, limit) { 
  ...
  }
\end{lstlisting}

then the function may be invoked in several ways, for example

\begin{lstlisting}
  fun1(d, df, TRUE, 20)
  fun1(d, df, graph = TRUE, limit = 20)
  fun1(data = d, limit = 20, graph = TRUE, data.frame = df)
\end{lstlisting}

    \item All of them are equivalent (cf.\ positional matching and keyword
      matching)
  \end{itemize}
\end{frame}


\begin{frame}[fragile]{Defaults}
  \begin{itemize}
    \item In many cases, arguments can be given commonly appropriate
      default values, in which case they may be omitted altogether from the
      call

\begin{lstlisting}
  fun1 <- function(data, data.frame, graph = TRUE, limit = 20) {
    ...
  }
\end{lstlisting}

    \item It could be called as

\begin{lstlisting}
  ans <- fun1(d, df)
\end{lstlisting}

which is now equivalent to the three cases above, or as

\begin{lstlisting}
  ans <- fun1(d, df, limit = 10)
\end{lstlisting}

which changes one of the defaults
  \end{itemize}
\end{frame}

\begin{frame}{}
  \begin{block}{Exercise}
  \begin{itemize}
    \item Write a function in R that cumulatively sums up the values of
      vector $\mathbf{x} = (1~2~3~4 \dots 20)'$
    \item The result should look like: $\mathbf{y} = (1~3~6~10 \dots 210)'$
  \end{itemize}
  \end{block}
\end{frame}

\section[if/else]{Conditional execution}

\begin{frame}[fragile]{Conditional execution}
  \begin{itemize}
    \item When programming, a distinction of cases is often necessary for
      \begin{itemize}
        \item checking of arguments
        \item return of error messages
        \item interrupting a running process
        \item case distinction, e.\,g., in mathematical expressions
      \end{itemize}

    \item Conditional execution of code is available in R via


\begin{lstlisting}
if(expr_1) {
  expr_2
} else {
  expr_3
}
\end{lstlisting}

where \verb+expr_1+ must evaluate to a single logical value
  \end{itemize}
\end{frame}

\begin{frame}[fragile]{Conditional execution}
\begin{lstlisting}
x <- 5

# Example 1
if(!is.na(x)) y <- x^2 else stop("x is missing")

# Example 2
if(x == 5){   # in case x = 5:
  x <- x + 1  # add 1 to x and
  y <- 3      # set y to three
} else        # else:
  y <- 7      # set y to seven

# Example 3
if(x < 99) cat("x is smaller than 99\n")
\end{lstlisting}
  \nocite{Ligges2008}
\end{frame}

\begin{frame}[fragile]{Conditional execution}
\begin{lstlisting}
## Vectorized version with ifelse() function

# Example 1
ifelse(x == c(5, 6), c("A1", "A2"), c("A3", "A4"))

# Example 2
x <- -2:2
ifelse(x < 0, -x, x)
\end{lstlisting}
\end{frame}

\begin{frame}{}
  \begin{block}{Exercise}
  \begin{itemize}
 \item Implement the following function in R:
\[
f(x) =
\begin{cases}
 -1 & \text{if } x < 0,\\
  0 & \text{if } x = 0,\\
  1 & \text{if } x > 0.
\end{cases}
\]
  \end{itemize}
  \end{block}
\end{frame}

\section[Loops]{Loops}

\begin{frame}{Loops}
  \begin{itemize}
    \item Loops are necessary to execute repeating commands
    \item Especially for simulations, loops are often used
    \item In this case, the same functions or commands are executed for
      different random numbers
    \item There are \texttt{for()} and \texttt{while()} loops for repeated
      execution
    \item The most simple ``loop'' is \texttt{replicate()}
  \end{itemize}
\end{frame}

\begin{frame}[fragile]{Loops}
\begin{lstlisting}
## Example for central limit theorem

y <- runif(100)     # Draw random numbers
hist(y)

x <- replicate(1000, mean(runif(100)))
hist(x)
\end{lstlisting}
\end{frame}

\begin{frame}[fragile]{Loops}
\begin{lstlisting}
i <- 0
repeat{
  i <- i + 1          # Add 1 to i
  if (i == 3) break   # Stop if i = 3
}

while (i > 1) {
  i <- i - 1          # As long as i > 1, subtract 1
}
\end{lstlisting}
\end{frame}

\begin{frame}[fragile]{Loops}
\begin{lstlisting}
x <- c(3, 6, 4, 8, 0)     # Vector of length 5
for(i in x)
  print(sqrt(i))

for(i in seq_along(x))    # Same using indices
  print(sqrt(x[i]))

for(i in seq_along(x)){   # For all i [1,2,3,4,5]'
  x[i] <- x[i]^2          # square ith element of x
  print(x[i])             # and show it on monitor
}

x^2                       # BETTER
\end{lstlisting}
\end{frame}

\begin{frame}{}
  \begin{block}{Exercise}
    \begin{itemize}
    \item Create a vector $\mathbf{x} = (3~5~7~9~11~13~15~17)'$ with a
    for-loop\\ Tip: Use the formula $n\cdot2 + 1$
    \item Implement two different methods:
    \begin{enumerate}
    \item Allocate memory: Start with a vector of zeros and the correct
      length and replace its elements iteratively
    
    \item Growing: Start with a \texttt{NULL} object and iteratively add new
      results
    \end{enumerate}
    Tip: The first method is more efficient, especially for long vectors
    \end{itemize}
  \end{block}
\end{frame}


\section[apply()]{Avoiding loops}

\begin{frame}{Avoiding loops}
  \begin{itemize}
    \item The \texttt{apply()} family of functions may be used in many
      places where in traditional languages loops are employed
    \item Using vector based alternatives is usually much faster in R
   % \item Matrices and arrays: \texttt{apply()}
   % \item Data frames, lists and vectors: \texttt{lapply()} and
   %   \texttt{sapply()}
   % \item Group-wise calculations: \texttt{tapply()}
  \end{itemize}
  \vspace{.2cm}
  \begin{center}
    \begin{tabular}{ll}
      \hline
      Matrices and arrays: & \texttt{apply()} \\
      Data frames, lists and vectors: & \texttt{lapply()} and
      \texttt{sapply()} \\
      Group-wise calculations: & \texttt{tapply()} \\
      \hline
    \end{tabular}
  \end{center}
\nocite{Ligges2008}
\end{frame}


\begin{frame}{\texttt{apply()}}
  \begin{itemize}
    \item \texttt{apply()} is used to work vector-wise on matrices or
      arrays
    \item Appropriate functions can be applied to the columns or rows
      of a matrix or array without explicitly writing code for a loop
  \end{itemize}
\end{frame}


\begin{frame}[fragile]{\texttt{apply()}}
\begin{lstlisting}
X <- matrix(c(4, 7, 3, 8, 9, 2, 5, 6, 2, 3, 2, 4), nrow = 3, ncol = 4)

# Calculate row maxima
res <- numeric(nrow(X))
for(i in 1:nrow(X)){
  res[i] <- max(X[i,])
}

# or:
apply(X, 1, max)  # Maximum for each row
apply(X, 2, max)  # Maximum for each column
\end{lstlisting}
\end{frame}

\begin{frame}{\texttt{lapply()} and \texttt{sapply()}}
  \begin{itemize}
    \item[lapply] Using the function \texttt{lapply()} (\texttt{l} because the
      value returned is a list), another appropriate function can be
      applied element-wise to other objects, e.\,g., data frames,
      lists, or simply vectors
      \begin{itemize}
        \item[$\to$] The resulting list has as many elements as the original
          object to which the function is applied
      \end{itemize}
      \vspace{.2cm}
    \item[sapply] Analogously, the function \texttt{sapply()} (\texttt{s} for
      simplify) works like \texttt{lapply()} with the exception that it
      tries to simplify the value it returns
      \begin{itemize}
        \item[$\to$] It might become a vector or a matrix
      \end{itemize}
  \end{itemize}
  \vfill
\end{frame}

\begin{frame}[fragile]{\texttt{lapply()} and \texttt{sapply()}}
\begin{lstlisting}
L <- list(x = 1:10, y = 1:5 + 0i)
lapply(L, mean)       # Keep list with data type
sapply(L, mean)       # Create vector, same data type

sapply(iris, class)   # Work column wise on data frame
\end{lstlisting}
\end{frame}

\begin{frame}{\texttt{tapply()}}
  \begin{itemize}
    \item \texttt{tapply()} (\texttt{t} for table) may be used to do
      group-wise calculations on vectors
    \item Frequently, it is employed to calculate group-wise means
  \end{itemize}
\end{frame}

\begin{frame}[fragile]{\texttt{tapply()}}
\begin{lstlisting}
data(Oats, package = "nlme")
with(Oats, tapply(yield, list(Block, Variety), mean))

data(warpbreaks)
tapply(warpbreaks$breaks, warpbreaks$tension, sum)
tapply(warpbreaks$breaks, warpbreaks[ , -1], mean)
\end{lstlisting}
\end{frame}

\begin{frame}{}
  \begin{block}{Exercise}
  \begin{itemize}
    \item Load the iris data set into R using \texttt{data(iris)}
    \item Write a for-loop to calculate the means for the dependent
      variables (columns~1~to~4)
    \item Think of as many vector based alternatives as possible to avoid
      this loop and calculate the column means
  \end{itemize}
  \end{block}
\end{frame}

\section[Random numbers]{Random number generation}

\begin{frame}{Random number generation}
  \begin{itemize}
    \item Most distributions that R handles have four functions
    \item There is a root name, e.\,g., the root name for the normal
      distribution is \texttt{norm} 
    \item This root is prefixed by one of the letters
      \texttt{p}, \texttt{q}, \texttt{d}, \texttt{r}
  \end{itemize}

  \vspace{.2cm}
  \begin{center}
{\renewcommand{\arraystretch}{1.5}
  \begin{tabular}{c|l}
    \texttt{p} & probability: the cumulative distribution function
    (CDF)\\
    \hline
    \texttt{q} & quantile: the inverse CDF\\
    \hline
    \texttt{d} & density: the probability (density) function (PDF)\\
    \hline
    \texttt{r} & random: a random variable having the specified
    distribution\\
  \end{tabular}
}
  \end{center}
\end{frame}


\begin{frame}{Random number generation}
  \begin{itemize}
    \item See \texttt{?Distributions} for a list of distributions or the
      CRAN task view \url{https://cran.r-project.org/view=Distributions}
    \item The random number generator in R is {\em seeded}: Upon restart
      of R, new random numbers are generated
    \item To replicate the results of a simulation, the seed (starting
      value) can be set explicitly
  \end{itemize}
\end{frame}


\begin{frame}[fragile]{Random number generation}
\begin{lstlisting}
# Examples
rnorm(10)     # Draw from standard normal distribution
rpois(10, 1)  # Draw from Poisson distribution

# Sampling with or without replacement from a vector
sample(1:5, size = 10, replace = TRUE)

# Set seed
set.seed(1223)  # On each run, random numbers will be identical
runif(3)
\end{lstlisting}
\end{frame}

\begin{frame}{}
  \begin{block}{Exercise}
  \begin{itemize}
    \item Create a data frame with four variables and 20 observations:
      \begin{enumerate}
        \item $X \sim N(\mu=100, \sigma^2=15^2)$
        \item $Y \sim Bin(n=10, p=0.2)$
        \item $Z \sim Pois(\lambda=1)$
        \item $S = X + Y + Z$
      \end{enumerate}
  \end{itemize}
  \end{block}
\end{frame}

\section[Data frames]{Data frames}

\begin{frame}{The fundamental data structure: data frames}
  \begin{itemize}
    \item Data frames are lists that consist of vectors and factors of
      equal length
    \item The rows in a data frame refer to one unit (observation or
      subject)
    \item For longitudinal data they can be reshaped between the long and
      the wide data format
  \end{itemize}
  \vfill
\end{frame}

\begin{frame}[fragile]{The fundamental data structure: data frames}
\begin{lstlisting}
# Creating a data frame
id <- factor(paste("s", 1:6, sep = ""))
weight <- c(60, 72, 57, 90, 95, 72)
height <- c(1.75, 1.80, 1.65, 1.90, 1.74, 1.91)
dat <- data.frame(id, weight, height)

# Variables in a data frame are extracted by $
dat$weight
\end{lstlisting}
  \nocite{Venables2022}
\end{frame}


\begin{frame}[fragile]{Working with data frames}

\begin{lstlisting}
# Frequently used functions (not only) for data frames

dim(dat)      # Show number of rows and columns

names(dat)    # Variable names

View(dat)     # Open data viewer; might be of little 
              # help for large data sets

plot(dat)     # Pairwise plots

str(dat)      # Show variables of dat

summary(dat)  # Descriptive statistics
\end{lstlisting}
\end{frame}


\begin{frame}[fragile]{Indexing variables}

\begin{lstlisting}
weight[4]           # 4th element
weight[4] <- 92     # Change 4th element
weight[c(1, 2, 6)]  # Elements 1, 2, 6
weight[1:5]         # Elements 1 to 5
weight[-3]          # Without element 3
# See ?Extract

# Indices may be logical.
weight[weight > 60]
weight[weight > 60 & weight < 80]
height[weight > 60 & weight < 80]
\end{lstlisting}

Logical expressions may contain AND~\texttt{\&}, OR~\texttt{|},
  EQUAL~\texttt{==},\\ NOT EQUAL~\texttt{!=}, and \texttt{<}, \texttt{<=},
  \texttt{>}, \texttt{>=}, \texttt{\%in\%}.
\end{frame}


\begin{frame}[fragile]{Indexing data frames}

\begin{lstlisting}
dat[3, 2]    # 3rd row, 2nd column
dat[1:4, ]   # Rows 1 to 4, all columns
dat[, 3]     # All rows, 3rd column

dat[dat$id == "s2", ]   # All observations of s2
dat[dat$weight > 60, ]  # All observations above 60kg
\end{lstlisting}
\end{frame}

% \begin{frame}[fragile]{Combining data frames}
% \begin{lstlisting}
% # Data frames having the same column names
% dat <- rbind(dat1, dat2, dat3)
% 
% # Merge by ID
% dat <- merge(dat1, dat2, by = "id", all = TRUE)
% \end{lstlisting}
% \end{frame}

\begin{frame}{}
  \begin{block}{Exercise}
    \begin{itemize} 
      \item Create a data frame with two independent variables: {\it Hand}
        with levels ``right'' and ``left'' and {\it Condition} with levels
        1, 2, 3, 4, and 5
      \item Simulate reaction times for 50 subjects; assume
        reaction time is normally distributed with $RT \sim
        N(\mu=400,\sigma^2=625)$
      \item There are 10 subjects in each condition
      \item Use functions \texttt{str()} and \texttt{summary()} on your
        data frame:\\
        What does the output tell you?
    \end{itemize}
  \end{block}
\end{frame}


\appendix

%\begin{frame}[allowframebreaks]{References}
\begin{frame}{References}
%\renewcommand{\bibfont}{\small}
  \printbibliography
\vfill
\end{frame}

\end{document}

